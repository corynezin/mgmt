\section{Introduction}
{\narrower
``In many ways, managing a large computer programming project is like managing any other large undertaking--in more ways than most programmers believe.  But in many other ways it is different--more ways than most professional managers expect.''
\par}

\medskip
\noindent
As Brooks says, management of software projects is very similar to management of any other project, but not so similar that management decisions should be identical.  Therefore the book, and this summary will not focus on the aspects of management that are similar, but rather those that are unique to the field.

The book is not scientific in nature, as it simply reflects the personal veiws of the author, which were developed during his time at IBM.  Brooks explains that the main project he worked on, OS/360, had both successes and failures in terms of its final design. The product was late, bloated, and much more expensive than estimated.  After several iterations, however, it was reliable, efficient, and versatile.

The author says that the book is ``a belated answer to Tom Watson's probing questions as to why programming is hard to manage.''  The book does not contain the knowledge and experience of Brooks alone, but also the thoughts and ideas from several other managers in the field with whom Brooks has conversed with.  

Brooks compares Large-system programming to a tar pit.  The harder you struggle, and the more resources you devote to getting out, the more you are enveloped.  It seems counterintuitive that some of most important work, for example the origins of Apple and Microsoft, were completed in garages by small teams.  However, this is a common theme throughout the book, that efficiency decreases with man power.  Yet large teams dedicated to increasingly large products are necessary because there is a giant cost differential between small, important programs that solve some specific landmark problem, and large general programs which work across the board for many consumers.


